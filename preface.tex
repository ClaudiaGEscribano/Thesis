\chapter*{Preface\label{cha:preface}}

\epigraphfontsize{\small\itshape}
\epigraph{''Habiendo accedido tarde a los estudios, sentía ahora urgencia por estudiar. A veces, inmerso en sus libros, le venía a la cabeza la conciencia de todo lo que no habñia leído y la serenidad con la que trabajaba se hacía trizas cuando caía en la cuenta del poco tiempo que tenía en la vida para aprender tantas cosas, para aprender todo lo que tenía que saber''}{--- \textup{Stoner}}

Applied science has made societies evolved trough the linked between the basic science and the 'ingenios' that implement the structural knowdlege to the individual dailylife. The bridge between the fundamental studies and its evolution or transformation into something applied (useful) for the societies development is been a matter of interpretation or 'translation' (so interpreters/translator) that have been able to find the accurate language to exchange this knowdlege.

% Esta transversalidad entre las disciplinas básicas de la física, la química o la biología con distintas acepciones prácticas, la mayoría de ellas recogidas bajo el paraguas de las ingenierías, desemboca ineludiblemente en una transformación de lo abstracto en lo tangible con la premisa o el ideal de mejorar el bienestar de las sociedades presentes y futuras.

Transversality between the fundamental disciplines of physiscs, chemistry or biology and their different applied branches, most of them under the umbrella of ingenieering, end unavoidably into a transformation from the abstract
ion to the tanglible with the premise or ideal of improving wellbeing of present and future societies.

% Es sin embargo muy probable, que esta concepción de la aplicación científica haya desembocado en el mayor problema al que la humanidad tiene que hacer frente desde su existencia, el cambio climático. Siendo este antropocentrismo parte del problema, y no de la solución. Jorge Wasenberg escibe en su libro, ``el pensador intruso'' en referencia a la evolución de la ciencia y el progreso subyacente lo siguiente:

However, it is very likely that this conception of the scientific application had lead to the most challeging problem that humanity has to face from its existance, climate change, being this anthropocentrism part of the problem and not of the solution.

In his book ``El pensador intruso'', Jorge Wasengber wrote about the evolution of science and the underlying progress what follows: 

'There are above all two vices that tend to set 'pre-cooked' ideollogy into science. First is base on different kinds of anthropocentrism and consist in put the knowdlege subjet into the cosmos's center. The history of knowdlege is the witness: each time we 'barremos' the 'I' from the spot, the knowdlege progresses and only because of it.'
% ``Existen sobre todo dos vicios que tienden a inyectar ideología precocinada en la ciencia. Una de ellas se basa en las distintas formas de antropocentrismo y consiste en situar instintivamente al sujeto del conocimiento en el centro del cosmos. La historia del conocimiento es testigo: cada vez que barremos el Yo del centro del escenario el conocimiento avanza, y avanza sólo por ello.''

% Sin caer en lo pretencioso, este trabajo contribuye a la imbricación entre la ciencia del cambio climático y la tecnología renovable de producción eléctrica con mayor potencial a día de hoy, la solar fotovoltaica.

