\part{Discussion \& Conclusion}

%%%%%%%%%%%%%%%%%%%%%%%%%%%%%%%%%%%%%%%%%%%%%%%%%%%%%%%%%%%%%%%

% \chapter{Discussion}

% % The work presented in this manuscript analysed the long-term characteristics of solar resource and the PV potential over the Euro-Mediterranean area. Three different approaches are adopted in order to address different issues related to that features.

% % * A discutir: el papel del aumento de temperatura en la producción fotovoltaica
% * lack of data to evaluate different simulations.
% * First time a sensitivity test with/without AOD is used with a regional climate models -> is an step foreward. Not only first direct effect but also indirect effects????

% \subsubsection{Regionalization over the IP}

% \begin{itemize}
% \item ¿qué aporta realmente la distribución espacial considerando que la variabilidad interanual es baja?
% \item ¿puede utilizarse la misma distribución para escalas más pequeñas? ¿habría que mostrarlo?   
% \item ¿será esta distribución igual en el futuro?
% \end{itemize}  
% \subsubsection{Impact of aerosols over the Mediterranean area}

% \begin{itemize}
% \item ¿Pueden utilizarse realmente los modelos regionales considerando su bias en radiación y nubosidad?
% \item ¿Cómo se mide la incertidumbre en las estimaciones que hemos hecho?
% \end{itemize}

% * Daily AOD is better than monthly AOD because most of the processes driving aerosols occur at synoptic scales spanning periods of onlu a few days. MOnthly AOD can lead to an understimation. (2 referencias de Gueymard para añadir a la introducción o a la discusión)

% \subsubsection{Future projections of PV potential under climate change scenarios}

% \begin{itemize}
% \item ¿Cómo afecta el aumento de temperatura a la eficiencia de la producción fotovoltaica?
% \end{itemize}

\chapter{Discussion and Conclusion\label{Conclusion}}

The main results of this work have shown that solar resource and PV production variability features is wide across different time and spatial scales deserving attention.  

In the three results chapters included in this work, the response to each specific question emerged from the base scientific issue that has been investigated. Long-term variability problems, that have not been widely considered among the literature, are studied here. Although there is that common place for the three research items, each one can be discussed individually and gives its own conclusions that are summarized below.

As the same time that some discussion is needed about the research undertaken, there are also some perspectives that arise from the results. These ones are also considered in a latter section.

% In first place, a methodology is applied to an electrically coherent area, the Iberian Peninsula, which means that this area can be considered as a 'unit' in a wider electrical system. 

% La variabilidad espacio-temporal del potencial fotovoltaico y del recurso solar en la zona Euro-Mediterránea en escalas climáticas está influenciada por variaciones de baja frecuencia en la nubosidad pero cambios en la concentración de aerosoles atmosféricos se ha demostrado que no son despreciables. Los cambios en AOD debido al aumento y disminución históricos en la concentración de aerosoles de tipo antropogénico influyen en el potencial fotovoltaico principalmente en áreas de Europa Central. Además la estacionalidad de los aerosoles naturales impacta en la variabilidad espacial de la productividad.

% Los modelos climáticos regionales se demuestran una herramienta necesaria para poder estimar la influencia de los aerosoles en las variables implicadas en el recurso fotovoltaico, puesto que permiten investigar los cambios debido

% Bajo condiciones de cambio climático, es necesario conocer si el recurso solar o el potencial fotovoltaico va a cambiar, debido a que se proyecta un aumento de temperatura y un cambio en los patrones de nubosidad, con un desplazamiento hacia el norte. En este aspecto, estimar la evolución de los aerosoles se convierte en un factor clave para conocer la evolución del recurso solar. La mayoría de los modelos y simulaciones disponibles en EURO-CORDEX no consideran una evolución temporal de los aerosoles, lo que lleva a pensar que un ensemble de modelos para estimar el potencial fotovoltaico futuro no es reliable.

\subsubsection{Clustering analysis over the IP}

In chapter 5 a multi-step scheme is applied to the Iberian Peninsula. It has been considered as a coherent area in the European electrical grid and energy system, because of its almost isolated character, from a physical and electrical point of view. In addition, the IP is considered as a significant case of study because of its wide variety of climates in a relatively small area, as it was remarked in the corresponding chapter. The scientific questions addressed through the clustering methodology can be summarized in three:

\begin{itemize}
  
\item In first place it is analyzed if there is an optimum spatial distribution of clusters able to explain variability of solar resource within the Iberian Peninsula.

\item Secondly, which are the main characteristics grouped together in that spatial distribution.

\item Finally, if there is spatial complementarity between sub-areas and if it can be studied with this method.

\end{itemize}

The scheme presented in \ref{cha:multi} allows to characterize an area according to its solar resource variability which systematize the inter-comparison of different areas and that can be applied to different time scales, different resources or different areas.

When the regionalization is applied to the Iberian Peninsula, main climatic features are captured by the method and the interannual variability can be analyze and spatially compared. In general, the CV of yearly productivity for different clusters among the IP goes from very low variability, 2$\%$, to around 5$\%$ which means that it is roughly stable in that time scales, although differences between areas are clear and the interannual variability of monthly series is clearly higher. The CV increases with the change in the tracking system, being the two axis tracking type more sensitive to changes in solar radiation. Due to that, differences between clusters are higher in that case.

This chapter shows that there is an spatial configuration of clusters that can define different variability modes over the Iberian Peninsula, which can be useful for further analysis. Also, this spatial analysis allows to investigate different temporal scales where spatial complementarity can arise. %This result can be applied to different resources as well as a combination of different technologies that can lead to a further research.


%** the resource is rather stable over the area, in time and in space for the interannual scales for yearly values, but the method is able to configurate an spatial distribution of clusters which is very useful for detect even small changes in the overall behaviour.

%There are many possibilities deriving from the methodology applied in this chapter. As it has been previously commented, the same methodology can be applied to different resources and time scales but in order to make concrete the most promising one 
\subsubsection{Impact of aerosols over the Mediterranean area}

The results of chapter 7 lead to the conclusion that not only variations in cloudiness are important for long-term solar radiation variability, also aerosols content affects in different ways over the area. Within this chapter, the answer to the next questions can be find:

\begin{itemize}
  
\item In first place, it wants to be known if aerosols have a significant impact on photovoltaic production over the Euro-Mediterranean area.
\item Secondly, how this aerosols influence the spatio-temporal variability, from seasonal to multidecadal scales, in actual climate conditions of photovoltaic production.
\item Third, the use of climate models for renewable energy studies and renewable energy assessment can be evaluated indirectly from this chapter.
\end{itemize}

Past observed trends of SSR over Europe have been simulated using a regional climate model. Only when an accurate aerosols dataset is included in the simulation, that takes into account sulfates trends over the area, the model is able to reproduce the increase in SSR since the 80's.

The multi-decadal simulation of SSR is used to quantify the impact of that trend in a simulated PV power plant. The results show that the brightening period over Europe would lead to an increase in yearly production of more than 10$\%$ in some areas of Central Europe. It means a big impact in a potential PV project over the area.

A sensitivity test has evaluated the impact of aerosols in the interannual and seasonal scale. It has not been observed a significant impact on the former but indeed high spatial and seasonal variability is observed.

%{\color{red} Past SSR over Europe simulated with a regional climate model is only able to follow the observed trend if an aerosol dataset is used to this purpose. In this context, these simulations are used in order to quantify the impact that these trends would have in a simulated PV power plant. The results show that the brightenning period over Europe would lead to an increase in yearly production of more than 10$\%$ in some areas of Central Europe. This means a big impact in a potential PV project over the area. Also, seasonal variability of PV productivity in actual climate conditions have a wide impact accross the area.}

The added value of simulations including aerosols has been demonstrated with the sensitivity test applied in this chapter. Despite the limitations of RCMs, is being illustrated that they are an useful tool in order to go further in the understanding of solar radiation and aerosols interactions.

%in this chapter illustrates that regional climate models are useful to these kind of experiments, despite of the limitations discussed in previous sections, in order to help in the knowdlege of solar radiation and aerosols interactions. The added value of simulations including aerosols dataset for the estimation of PV production is clear for the whole domain under study. Also it is shown that PV production data can be reproduce using simulations with aerosols for at least two real PV plants location.

It is important to remark that due to the difficulties to obtain real PV data, it is difficult to validate the complete modeling chain approach. Nevertheless, each step of the methodology followed is being previously validated successfully.

In order to reduce uncertainty in the PV estimations, it would be also possible to reduce solar irradiation bias through a correction methodology. It is noteworthy that we have seen (and it has been shown by other authors) that inclusion of aerosols highly improve solar irradiation at the surface but also, some bias correction can be forward applied and evaluated in order enhance solar radiation as input of the modeling chain.  

% In order to go further in the use of RCMs for energy resource assessment, being able to compare with real data is one of the 'bottle necks'. However, in order to reduce uncertainty in the PV estimations, it is possible to reduce solar irradiation bias. In first place, we have seen (and it has been shown by other authors) that inclusion of aerosols highly improve solar irradiation at the surface but also, some bias correction can be forward applied and evaluated in order to be used as input of the modelling chain.  


\subsubsection{Future projections of PV potential under climate change scenarios}

The increasing concern about availability of renewable resources under future climate conditions has motivated recent research applying climate models to evaluate possible changes in different resources. 

Surface solar radiation under climate change scenarios has been investigated in different works as well as the PV potential future projections. However, the number of research papers dedicated to future renewable energy under climate change is rather limited.

The discrepancy between global climate models and regional climate models in future SSR over Europe is an issue that deserves attention. In chapter 7, the relationship between different climate projections and its aerosols representation is investigated in order to answer the next questions:

\begin{itemize}
\item Due to the importance that possible changes in solar resource have for planning renewable energy activities, How are the estimations of future PV potential over the Euro-Mediterranean area under climate change scenarios?
\item Is the role of aerosols and its evolution in future projections important to understand discrepancies between GCMs and RCMs' SSR projections and photovoltaic potential over that area?

\end{itemize}

  Regional climate simulations from the EURO-CORDEX ensemble are used in the analysis of PV potential. The results show that the use of evolving aerosols in future projections is key to the evaluation of future PV production anomalies. RCMs simulations using time evolving aerosols reproduce an increase in photovoltaic productivity over Europe, which was observed by GCMs and reverse the trend with respect to the rest RCM simulations. This show that including aerosols evolution in the future projections is important in order to narrow up uncertainties among simulations of different models. 

  This result is different from previous studies because is focused on the selection of simulations according to its aerosols representation and discard the ensemble overview.
  
  For projections including aerosols, the PV potential anomaly for the mid XXI century depends on the area, although higher changes are in central and Southern Europe. This pattern is related to different scenarios of aerosols that project a decrease in anthropogenic aerosols around those areas.

  The range of changes varies from one model to another, a result that is important not only because of the cautious to take care giving messages to the solar industry, but also because it points out the way to follow in future works that should quantify the aerosols forcing impact for every model. The projected FPS inside the EURO-CORDEX framework is an ongoing exercise that systematize the RCM's simulations applying different sensitivity test with and without different aerosols dataset and it can help to better understand how aerosol evolution affects not only surface solar radiation but climate.

\chapter{Perspectives\label{perspectives}}

Some perspectives derive from the present work and can lead future research in the field.

On the first place, the methodology applied in chapter 5 can be easily applied to different technologies, areas or time-scales. It would be interesting to apply the methodology to a combination of resources, in order to obtain the spatial distribution that better complement power from different technologies like PV and wind or hydropower. An step forward would be take into account real power plants locations, although obtaining this kind of information is challenging. Once the spatial distribution of power plants is considered real applications to improve efficiency of planning and operation activities can be done in for the targeted areas.

On the other hand, it would be interesting to investigate if the optimum partition changes in time and under climate change conditions, which means changes in the variability of solar resource. In that case, it would be necessary again the use of climate models. The same methodology could be applied for different RCPs scenarios. 

When the study moves towards the use of climate models for the research of solar resource and PV potential, there is a wider spectrum of opportunities to go for further analysis.

\begin{itemize}
\item First, different sensitivity tests can be designed in order to understand different processes affecting solar irradiation.
\item Future evolution of RCMs with prognostic schemes will allow us to design better experiments to understand the role of aerosols in variability for other time scales. Extreme events can be modeled and further investigated. For instance, the reported extreme dust outbreak that has been reported in Germany for the 4Th of April in 2014, is an interesting case of study to test on the first place if models are able to reproduce the observed event and afterward, to estimate the impact of dust in PV production (from local to regional scales) in that kind of episodes.  
\item The non homogeneous way of representing aerosols in regional climate simulations is an open research field whose contributions are straightforward. From the \textit{Flagship Pilot Study}, FPS, launched in the core of EURO-CORDEX community, it is expected to contribute in the understanding of aerosols and climate interactions over the Euro-Mediterranean area. From the energy perspective, those contributions would lead to a better advise and energy projections for the energy industry and the climate services.
\item It is also important to notice that different areas around the world can also evolve in similar ways than Europe. For instance, the observed decrease in anthropogenic aerosols emission in Europe will probably occur in other areas in the future. It means that better estimation of future PV potential in Europe, as well as the quantification of the impact of brightening trends can also contribute for a better development of the PV technologies in other areas.  