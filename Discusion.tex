\part{Discussion \& Conclusion}

%%%%%%%%%%%%%%%%%%%%%%%%%%%%%%%%%%%%%%%%%%%%%%%%%%%%%%%%%%%%%%%

\chapter{Discussion}

% The work presented in this manuscript analysed the long-term characteristics of solar resource and the PV potential over the Euro-Mediterranean area. Three different approaches are adopted in order to address different issues related to that features.

% * A discutir: el papel del aumento de temperatura en la producción fotovoltaica
* lack of data to evaluate different simulations.
* First time a sensitivity test with/without AOD is used with a regional climate models -> is an step foreward. Not only first direct effect but also indirect effects????

\subsubsection{Regionalization over the IP}

\begin{itemize}
\item ¿qué aporta realmente la distribución espacial considerando que la variabilidad interanual es baja?
\item ¿puede utilizarse la misma distribución para escalas más pequeñas? ¿habría que mostrarlo?   
\item ¿será esta distribución igual en el futuro?
\end{itemize}  
\subsubsection{Impact of aerosols over the Mediterranean area}

\begin{itemize}
\item ¿Pueden utilizarse realmente los modelos regionales considerando su bias en radiación y nubosidad?
\item ¿Cómo se mide la incertidumbre en las estimaciones que hemos hecho?
\end{itemize}

* Daily AOD is better than monthly AOD because most of the processes driving aerosols occur at synoptic scales spanning periods of onlu a few days. MOnthly AOD can lead to an understimation. (2 referencias de Gueymard para añadir a la introducción o a la discusión)

\subsubsection{Future projections of PV potential under climate change scenarios}

\begin{itemize}
\item ¿Cómo afecta el aumento de temperatura a la eficiencia de la producción fotovoltaica?
\end{itemize}

\chapter{Conclusion\label{Conclusion}}

Main results of this work show that variability in solar resource and PV production is wide and transversal to different time and spatial scales deserving attention.  

In the three results chapters included in this work there are different answers to the specific questions among the main topic that is to characterize variability of solar resource and photovoltaic potential over the Euro-Mediterranean area. Some long-term issues that has not been taken into account so frequently are studied here. Although there is this common place for the three research items each one has its own conclusions and are summarized below in different sections.

As the same time that some discussion is needed from the research undertaken and it was addressed previously, there are also some persepective questions that arise from the results. These ones are also considered in this last section.

% In first place, a methodology is applied to an electrically coherent area, the Iberian Peninsula, which means that this area can be considered as a 'unit' in a wider electrical system. 

% La variabilidad espacio-temporal del potencial fotovoltaico y del recurso solar en la zona Euro-Mediterránea en escalas climáticas está influenciada por variaciones de baja frecuencia en la nubosidad pero cambios en la concentración de aerosoles atmosféricos se ha demostrado que no son despreciables. Los cambios en AOD debido al aumento y disminución históricos en la concentración de aerosoles de tipo antropogénico influyen en el potencial fotovoltaico principalmente en áreas de Europa Central. Además la estacionalidad de los aerosoles naturales impacta en la variabilidad espacial de la productividad.

% Los modelos climáticos regionales se demuestran una herramienta necesaria para poder estimar la influencia de los aerosoles en las variables implicadas en el recurso fotovoltaico, puesto que permiten investigar los cambios debido

% Bajo condiciones de cambio climático, es necesario conocer si el recurso solar o el potencial fotovoltaico va a cambiar, debido a que se proyecta un aumento de temperatura y un cambio en los patrones de nubosidad, con un desplazamiento hacia el norte. En este aspecto, estimar la evolución de los aerosoles se convierte en un factor clave para conocer la evolución del recurso solar. La mayoría de los modelos y simulaciones disponibles en EURO-CORDEX no consideran una evolución temporal de los aerosoles, lo que lleva a pensar que un ensemble de modelos para estimar el potencial fotovoltaico futuro no es reliable.

\subsubsection{Regionalization over the IP}

In \ref{cha:multi} a multi-step scheme is applied to the Iberian Peninsula considered as a coherent area in the European electrical grid and energy system, because of its almost isolated character physical and electrical. Another important reason to select the IP as a significant case of study is its wide variety of climates in a relatively small area as it was remarked in the previous chapters. The scientific questions addressed trhough the clustering methodology can be summarized in three:

\begin{itemize}
  
\item In first place it is analysed if there is an optimum spatial distribution of clusters able to explain variability of solar resource among the Iberian Peninsula.

\item Secondly which are the main characteristics grouped together in that spatial distribution.

\item Finally, if there is spatial complementarity between sub-areas and if it can be studied with this method.

\end{itemize}

The scheme presented in \ref{cha:multi} allows to characterize an area according to its solar resource variability which systematize the inter-comparison of different areas and that can be apply different time scales, different resources or different areas.

When the regionalization is applied to the Iberian Peninsula main climatic features are captured by the mehtod and the interannual variability can be analyse and spatially compared. In general the CV of different clusters among the IP goes from very low variability, 2$\%$, to around 5$\%$ which means that it is roughly stable in that time scales, although differences between areas are clear and the interannual variability of each month is higher. The CV increases with the change in the tracking system, being the two axis tracking type more sensitive to changes in solar radiation. Due to that, differences between clusters are higher in that case.

This chapter shows that there is an spatial configuration of clusters that can define different variability modes over the Iberian Peninsula, which can be useful for further analysis as the ones commented in previous sections. Also, this spatial analysis allows to investigate different temporal scales where spatial complementarity can arise. This result can be applied to different resources as well as a combination of different technologies that can lead to a further research.

Future works can be developed considering real power plants locations and in different target areas, although it is a challenge to obtain this kind of information. Once the spatial distribution of power plants is considered real applications to improve efficiency of planning and operation activities can be done.

* Future: Over the IP some modes of variability has been detected (arturo sanchez-lorenzo) que favorece las intrusiones de polvo. La caracterización de estas situaciones, podría servir observar el impacto que estas situaciones tienen sobre cada uno de los clusters y su inter relación.  Esto permite conocer de manera general las zonas más afectadas por estas situaciones y generalizar en el espacio, lo que simplifica la toma de decisiones.

%** the resource is rather stable over the area, in time and in space for the interannual scales for yearly values, but the method is able to configurate an spatial distribution of clusters which is very useful for detect even small changes in the overall behaviour.

%There are many possibilities deriving from the methodology applied in this chapter. As it has been previously commented, the same methodology can be applied to different resources and time scales but in order to make concrete the most promising one 
\subsubsection{Impact of aerosols over the Mediterranean area}

The results of chapter 7 lead to the conclusion that not only variations in cloudiness are important for solar radiation variability, but also aerosols affect in different ways over the area. It can be found the answer to next questions:

\begin{itemize}
  
\item The scientific question adressed in this chapter try to answer if aerosols have an impact in photovoltaic production in the spatio-temporal variability from seasonal to multidecadal scales in actual climate conditions over the Mediterranean area.
\item For answering this section a RCM is used, which could help in a second order to see if these models are reliable to be use in renewable energy assessment. 
\end{itemize}

Past SSR over Europe simulated with a regional climate model is only able to follow the observed trend if an aerosol dataset is used to this purpose. In this context, these simulations are used in order to quantify the impact that these trends would have in a simulated PV power plant. The results show that the brightenning period over Europe would lead to an increase in yearly production of more than 10$\%$ in some areas of Central Europe. This means a big impact in a potential PV project over the area. Also, seasonal variability of PV productivity in actual climate conditions have a wide impact accross the area.

The sensitivity test applied in this chapter illustrates that regional climate models are useful to these kind of experiments, despite of the limitation discussed in previous sections, in order to help in the knowdlege of solar radiation and aerosols interactions. The added value of simulations including aerosols dataset for the estimation of PV production is clear for the whole domain under study. Also it is shown that PV production data can be reproduce using simulations with aerosols for at least two real PV plants location.

Due to the difficulties to obtain real PV data, the validation of the complete modelling chain becomes difficult. In order to go further in the use of RCMs for energy resource assessment, being able to compare with real data is one of the 'bottle necks'. However, in order to reduce uncertainty in the PV estimations, it is possible to reduce solar irradiation bias. In first place, we have seen (and it has been shown by other authors) that inclusion of aerosols highly improbe solar irradiation at the surface but also, some bias correction can be forward applied and evaluated in order to be used as input of the modelling chain.  


\subsubsection{Future projections of PV potential under climate change scenarios}

Surface solar radiation under climate change scenarios has been investigated in different works as well as the PV potential future projections. However, the number of research papers dedicated to future renewable energy under climate change is rather limited.

The discrepancy between global climate models and regional climate models in future SSR over Europe is an issue that diserves attention. Along chapter 7 the relationship between different climate projections and its aerosols representation is investigated answering next questions:

\begin{itemize}
\item Due to the importance that possible changes in solar resource have in planning renewable energy activities, How are the estimations of future PV potential over the Euro-Mediterranean area under climate change scenarios?
\item Is the role of aerosols and its evolution in future projections important to understand discrepancies between GCMs and RCMs' SSR projections and photovoltaic potential over that area?

\end{itemize}

  Regional climate simulations from the EURO-CORDEX ensemble are used in the analysis of PV potential. The results show that the use of evolving aerosols in future projections is key to the evaluation of future PV production anomalies. RCMs simulations using time evolving aerosols reproduce an increasing trend in photovoltaic productivity over Europe, which was observed by GCMs and reverse the trend with respect to the rest RCM simulations. This show that including aerosols evolution in the future projections is important in order to narrow up uncertainties among simulations of different models. 

  This result is different from previous studies because is focused on the selection of simulations according to its aerosols representation.
  
  For projections including aerosols, the PV potential anomaly for the mid XXI century depends on the area, although higher changes are in central and Southern Europe. This pattern is related to different scenarios of aerosols that project a decrease in anthropogenic aerosols around those areas.

  The range of changes varies from one model to another, a result that is important not only because of the cautious to take care giving messages to the solar industry but also because it points out the way to follow in future works that should quantify the aerosols forcing impact for every model. The projected FPS inside the EURO-CORDEX framework is an ongoing exercice that systematize the RCM's simulations applying different sensitivity test with and without different aerosols dataset and it can help to better understand how aerosol evolution affects not only surface solar radiation but climate.

