\chapter*{Agradecimientos\label{cha:agradecimiento}}

En el momento que escribo estas líneas, poco o nada queda de aquella que comenzó esta andadura hace ya algo más de cuatro años. Atrás han quedado la ilusión y la inocencia, barridas por el pragmatismo y la realidad de un trabajo diario solitario y absorbente, llevado a término en la presente forma gracias a una tenacidad y dedicación casi inagotable.

A pesar de la ausencia de un sentimiento profundo al llevar a término este trabajo, como probablemente esperaba que fuera en otro tiempo, la consciencia de cada uno de los pasos que me han llevado hasta aquí es plena, y el agradecimiento para todos y cada uno de los que han ayudado, en una u otra manera a que esto sea posible, es de justo reconocimiento.

Gracias a Miguel Ángel por darme la oportunidad de empezar esta tesis doctoral sin conocerme, sin ninguna referencia sobre mí y en una convocatoria contrarreloj solicitada en pleno verano. Espero que a día de hoy, no se arrepienta de aquella apuesta.\\

Cuento con los dedos de una mano los profesores que valieron la pena en mi paso por la universidad. El sistema ahoga a los profesores que quieren ser investigadores y a los investigadores que quieren ser profesores, perjudicando al alumnado que casi inerte recibe clases aburridas de profesores aburridos sobre cosas aburridas. Tuve que pasar por un posgrado de orientación profesional, para encontrar la esperanza que para entonces había desaparecido y volver a darme cuenta de que mi camino no pasaba por dedicar mis días a calcular el \textit{return of investment} en una hoja de c\{'}alculo. Gracias Oscar, porque tu trabajo y soporte incondicional te definieron, entonces y ahora, como un gran profesor. Gracias porque además, de manera transversal, me has proporcionado las herramientas y metodología de trabajo con las intentaré abrirme paso en la selva que se avecina. \\

La realidad de la investigación ha llevado a que sean unas cuantas las personas que han pasado por el laboratorio 0.14 dejándolo un poco huérfano con su marcha. Gracias, Marta. Tu carácter, tu disponibilidad infinita, tu amistad y tu comprensión hizo que aquellos primeros meses sean ahora un bonito recuerdo y ha ayudado a que los más difíciles sean una lección de la que aprender. Gracias a Raquel, a Vicky, a Noelia, a Jesús, a María. Gracias Juanje, porque siempre fuiste un paso por delante, y cada uno de tus consejos fueron valiosos para sobrevivir. Gracias Kike, el antagonismo entre tu optimismo exacervado y mi ofuscado carácter encontraron el equilibrio. Todo esto hubiera sido mucho más difícil sin tu apoyo. Gracias María, por estar cada d\{'}a para escucharme y apoyarme. 

\starbreak

Si en estos cuatro años tengo que agradecer algo, es sin duda la ayuda, el apoyo desinterado y en muchos casos la amistad, de todos los que compartieron tiempo conmigo en Toulouse.

Gracias Samuel por abrirme las puertas de vuestro grupo y hacerme sentir como una más. Gracias por tu ayuda, por cada uno de los consejos, por tu inagotable curiosidad y por cada una de tus lecciones. Es un privilegio poder haber compartido con vosotros ese tiempo y estaré siempre agradecida.

Gracias Pierre, por tu amabilidad, tu disponibilidad e inestimable ayuda. Gracias Marc, por regalarme tu tiempo que es excaso y hacer siempre las preguntas adecuadas.

Este trabajo nunca hubiera podido salir adelante sin vosotros.

No quiero olvidarme de agradecer a todos los que hicieron mi paso por aquella ciudad mas amable, dentro y fuera del trabajo. Sofía, Robin, Filippa, Françesca, Marie...gracias. De manera especial gracias a Danila. 

\starbreak

Por ultimo, estas lineas van dirigidas a mis padres porque siempre creyeron en mi un poco más de lo que debían y por enseñarme la riqueza de los libros y fomentar la curisidad que me ha llevado a ser quien soy. A mi hermana Lina.A Diego, por acompañarme.


\starbreak

A Diego. Gracias por acompañarme. 


