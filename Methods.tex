\chapter{Methods\label{cha:methods}}
\section{Clustering algorithm applied to climate data}
% The culstering algorithms are design for grouping together variables and recognise patterns between them that are difficult to see at first sigth. This algorithm were first applied in ... and they have grown quickly adapting to a highly data-driven world.

% Pattern recognition: extraer objetos y agruparlos en clases. Dependiendo de las distintas disciplinas, estos objetos pueden ser muy diferentes. Se utiliza el pattern recognition en el reconocimeiento de imágenes, de palábras, ayuda en el diagnóstico, o para el análisis de bases de datos en data mining. En este último, las aplicaciones van desde la biología a las finanzas o ciencias sociales y en un data-driven world, cada vez adquiere una importancia mayor, transformando los datos en conocimiento.

% En primer lugar, se definen las características que se van a emplear como medida de similitud/similaridad utilizada para la clasificación. Una vez que estas características están definidas, los algoritmos de clustering se encargan de agrupar  estas características.

En un mundo cada vez más movido por los datos, el reconocimiento de patrones se ha utilizado en  muchas disciplinas, desde la biología a las finanzas o las ciencias socialses, para extraer información relevante de los diferentes conjuntos de datos. Éstas técnicas, ayudan a conocer relaciones difíciles de extraer a simple vista, bien por el volumen de los datos, o por la complejidad de las mismas y la cantidad de variables involucradas. 



\subsection{Hierarchical and non-hierarchical methods}
\subsection{Principal components analysis and K-means}
\subsection{Validity index}
\subsection{Complete scheme}
\section{Simulating a photovoltaic system}
\section{Regional Climate Modeling simulations}
\section{Future projections and scenarios}
