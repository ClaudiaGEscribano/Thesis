% Este documento contiene el índice del manuscrito de la tesis.
%\documentclass[12pt,a4paper]{report}
%\usepackage{graphicx}
 
%\title{Introduction}
%\author{Claudia Guti\'errez}
%\date{ January 2018}


%\begin{document}

%\maketitle

%\tableofcontents{}

%%%%%%%%%%%%%%%%%%%%%%%%%%%%%%%%%%%%%%%%%%%%%%%%%%%%%%%%%%%%%%%%%%%%%%%%%%%%%%%%%%%%%%
\part{Introduction\label{cha:intro}}

%\section{Transversalidad}

% La ciencia aplicada o aplicación de la ciencia, ha evolucionado las sociedades a través de la conexión de los estudios de ciencia básica con los ingenios que implementaban el conocimiento estructural al día a día de los individuos. La construcción del puente entre los estudios fundamentales y su adaptación/evolución para el desarrollo de las sociedades, ha sido cuestión de una interpretación (y por lo tanto de intérpretes) capaz de econtrar el lenguaje adecuado para intercambiar este conocimiento.

%% Applied science has made societies evolved trough the linked between the basic science and the 'ingenios' that implement the structural knowdlege to the individual daylyligde. The bridge between the fundamental studies and its evolution or transformation into something applied for the societies development, has to be with the interpretation (so interpreters) that have been able to find the accurate languafe to exchange this knowdlege. %%

 % Esta transversalidad entre las disciplinas básicas de la física, la química o la biología con distintas acepciones prácticas, la mayoría de ellas recogidas bajo el paraguas de las ingenierías, desemboca ineludiblemente en una transformación de lo abstracto en lo tangible con la premisa o el ideal de mejorar el bienestar de las sociedades presentes y futuras. 

% Es sin embargo muy probable, que esta concepción de la aplicación científica haya desembocado en el mayor problema al que hacer frente desde.Siendo este antropocentrismo parte del problema, y no de la solución. Jorge Wasenberg escibe en su libro, ``el pensador intruso'' en referencia a la evolución de la ciencia y el progreso subyacente lo siguiente:

% ``Existen sobre todo dos vicios que tienden a inyectar ideología precocinada en la ciencia. Una de ellas se basa en las distintas formas de antropocentrismo y consiste en situar instintivamente al sujeto del conocimiento en el centro del cosmos. La historia del conocimiento es testigo: cada vez que barremos el Yo del centro del escenario el conocimiento avanza, y avanza sólo por ello.''

% Sin caer en lo pretencioso, este trabajo contribuye a la imbricación entre la ciencia del cambio climático y la tecnología renovable de producción eléctrica con mayor potencial a día de hoy, la solar fotovoltaica.

\chapter{Introduction}

\epigraphfontsize{\small\itshape}
\epigraph{''Begin at the beginning,'' the King said gravely, ''and go on till you
come to the end: then stop.''}{--- \textup{Lewis Carroll}, Alice in Wonderland}

\section{A changing world} 

Ideas a desarrollar:

* Un mundo en constante evolución.
* Cambio climático
* Transición energética global
* Cambios sociales/políticos.
* Migraciones

* Entrelazamiento de todas estas cuestiones.
* El conocimiento puro + el conocimiento aplicado: obligación de la ciencia a ser en la medida de lo posible aplicada a solucionar problemas de las distintas sociedades. Justicia social. Más si estos problemas han sido creados por el ser humano.


%Applied science has made societies evolved trough the linked between the basic science and the 'ingenios' that implement the structural knowdlege to the individual dailylife. The bridge between the fundamental studies and its evolution or transformation into something applied (useful) for the societies development is been a matter of interpretation or 'translation' (so interpreters/translator) that have been able to find the accurate language to exchange this knowdlege.

% Esta transversalidad entre las disciplinas básicas de la física, la química o la biología con distintas acepciones prácticas, la mayoría de ellas recogidas bajo el paraguas de las ingenierías, desemboca ineludiblemente en una transformación de lo abstracto en lo tangible con la premisa o el ideal de mejorar el bienestar de las sociedades presentes y futuras.

%Transversality between the fundamental disciplines of physiscs, chemistry or biology and their different applied branches, most of them under the umbrella of ingenieering, end unavoidably into a transformation from the abstraction to the tanglible with the premise or ideal of improving wellbeing of present and future societies.

% Es sin embargo muy probable, que esta concepción de la aplicación científica haya desembocado en el mayor problema al que la humanidad tiene que hacer frente desde su existencia, el cambio climático. Siendo este antropocentrismo parte del problema, y no de la solución. Jorge Wasenberg escibe en su libro, ``el pensador intruso'' en referencia a la evolución de la ciencia y el progreso subyacente lo siguiente:

%However, it is very likely that this conception of the scientific application had lead to the most challeging problem that humanity has to face from its existance, climate change, being this anthropocentrism part of the problem and not of the solution.

%In his book ``El pensador intruso'', Jorge Wasengber wrote about the evolution of science and the underlying progress what follows: 

%'There are above all two vices that tend to set 'pre-cooked' ideollogy into science. First is base on different kinds of anthropocentrism and consist in put the knowdlege subjet into the cosmos's center. The history of knowdlege is the witness: each time we 'barremos' the 'I' from the spot, the knowdlege progresses and only because of it.'
% ``Existen sobre todo dos vicios que tienden a inyectar ideología precocinada en la ciencia. Una de ellas se basa en las distintas formas de antropocentrismo y consiste en situar instintivamente al sujeto del conocimiento en el centro del cosmos. La historia del conocimiento es testigo: cada vez que barremos el Yo del centro del escenario el conocimiento avanza, y avanza sólo por ello.''

% Sin caer en lo pretencioso, este trabajo contribuye a la imbricación entre la ciencia del cambio climático y la tecnología renovable de producción eléctrica con mayor potencial a día de hoy, la solar fotovoltaica.

\section{Renewable Energies}

% La energía renovable es la forma de energ
Podemos definir el concepto físico de energía como la capacidad de un sistema para realizar un trabajo, clasificando las distintas formas de energía en dos grandes grupos: la energía cinética, relacionada con el movimiento del sistema, o la energía potencial, que atiende a la posición de mismo.\\

En esta clasificación fundamental encontramos que la energía mecánica, la energía electromagnética o la energía térmica son formas de energía que se engloban dentro de la energía cinética, mientras que la energía química (de la cual es un subtipo la energía nuclear) o la energía potencial gravitacional (como es el caso de las centrales hidroeléctricas) se encontrarían dentro de la energía potencial.\\

Es elemental para nuestro estudio delimitar la diferencia entre formas y fuentes de energía. Estas útimas son definidas como aquellas a partir de las cuales la energía 'útil', aquella que es directamente aplicable para su finalidad sea esta el movimiento, la producción de electricidad o los distintos procesos metabólicos en el ser humano, puede ser extraída directamente o mediante algun proceso de transformación. Así, los combustibles fósiles son una fuente de energía, puesto que su energía 'útil' debe ser extraída. Es decir, extraemos la energía química contenida en los combustibles fósiles y necesitamos ciertos procesos de transformación entre la energía potencial de una masa de agua elevada que termina pasando a través de una turbina para obtener electricidad. Algunos ejemplos de fuentes de energía son los combustibles fósiles, la energía solar o la biomasa.\\  

Aparece de manera natural la clasificación de estas fuentes de energía según su origen renovable, como la energía solar, eólica o geotérmica y no renovable como los combustibles fósiles. La energía renovable, es por lo tanto, aquella que tiene su origen en estas fuentes inagotables de energía. El sol, a pesar de su indiscutible finitud,  es considerado una fuente inagotable de energía debido a la diferencia entre la escala temporal de la vida humana y la vida de la estrella a la que nos refereimos, muchos órdenes de magnitud mayor.\\

El uso de la energía renovable ha acompañado el desarrollo de la humanidad desde tiempos remotos. Desde el uso de la biomasa para generar energía térmica para calentarse, hasta la utilización de la energía eólica transformada en energía mecánica en los molinos tradicionales o en la navegación. Sólo hacia la mitad del siglo 19, con la invención de la máquina de vapor, comenzó la utilización de los combustibles fósiles y con ello lo que se denominó la primera revolución industrial.\\

A pesar de constituir un periodo de gran desarrollo en términos tanto tecnológicos como de las sociedades y su bienestar, este lapso de tiempo es pequeño comparado con la historia de la humanidad. Además, este desarrollo, a pesar de haber conseguido que los niveles de vida alcancen los más altos estándares en nuestra historia, este hecho no ha sido ni mucho menos global ni 'isótropo' (homogéneo). \\


%La aparición de los combustibles fósiles comienza con la invención de la máquina de vapor, hacia la mitad del siglo 19. A pesar de constituir un periodo de gran desarrollo en términos tanto tecnológicos como de desarrollo de las sociedades y su bienestar, este lapso de tiempo es pequeño comparado con la historia de la humanidad. Además, este desarrollo, a pesar de haber conseguido que los niveles de vida alcancen los más altos estándares en nuestra historia, este hecho no ha sido ni mucho menos global ni 'isótropo'. 

%Puesto que en nuestro caso estaremos hablando en Es importante definir también la idea de fuente de energía. De la energía en sus diferentes formas no toda es útil en su estado original, sino que necesita ciertas transformaciones para extraer 'energía útil'. Por ejemplo, la energía química contenida en los combustibles fósiles necesita ser 'extraída' para ser utilizada

%A source from which useful energy can be extracted or recovered either directly or by means of a conversion or transformation process (e.g. solid fuels, liquid fuels, solar energy, biomass, etc.)

%Además de esta clasificación clásica, podemos atender a una clasificación de las formas de energía según su origen. De esta manera, encontraríamos las formas de energía de origen renovable y no renovable.

% *** Fuente o Forma de energía:
% Las \textbf{formas de energía} son: energía mecánica, térmica, química, energía eléctrica, etc. Las distintas formas de energía pueden ser o se agrupan en energía potencial o energía cinética.
% - La energía nuclear, es un subtipo de energía química.
% Las \textbf{fuentes de energía} pueden ser de tipo renovable o no renovable según su origen. Una fuente de energía es, por ejemplo, la energía eólica, porque es energía mecánica que puede ser transformada o puede ser origen de un otro tipo de energía secundaria/útil.
% * El uso de la terminología 'fuente de energía' es engañoso, puesto que por el principio ``la energía no se crea ni se destruye'' la energía no tiene un origen y no 'mana' de estas fuentes.
% ``A source from which useful energy can be extracted or recovered either directly or by means of a conversion or transformation process (e.g. solid fuels, liquid fuels, solar energy, biomass, etc.)''

%La energía renovable es aquella que se obtiene de fuentes inagotables de energía. El sol es considerado una fuente inagotable de energía a pesar de su indiscutible finitud debido a la diferencia entre la escala temporal de la vida humana y la vida de la estrella a la que nos refereimos, muchos órdenes de magnitud mayor. El uso de tecnologías renovables ha acompañado el desarrollo de la humanidad desde tiempos remotos. Desde el uso de la biomasa para generar energía térmica para calentarse, hasta la utilización de la energía eólica transformada en energía mecánica en los molinos tradicionales o en la navegación.


%El uso de los combustibles fósiles desde la primera revlución industrial ha permitido

% The evolution of renewable energies since its origin has been made in a context with two main drivers. In first place, a society concerned about environmental issues has to deal with 


% as well as interested in finding alternatives to traditional energy sources, Alternative energy generation sources have been evolving and have adquired an important role, mostly in the electricity sector, in the generation proccess
%\subsection{Context of renewable energies}
%\subsubsection{History and evolution}
%\subsection{Scenarios and future evolution of renewable energies}
%\subsection{Photovoltaic technology}
\section{Links between climate and renewable energy}


\section{Photovoltaic Energy}
%\section{Resource assessment}
\section{Climate change and the Mediterranean area}
\section{General scientific question: spatiotemporal behaviour of solar resource and photovoltaic production}
\section{Organization of the manuscript}
% \section{General scientific question: spatiotemporal variability of solar resource and photovoltaic production}
% \section{State of the art}
% \subsection{From short to long term varibility issues}
% \subsection{Solar radiation data measurements}
% \subsection{Satellite data}
% \subsection{Modelization of solar radiation}
% \section{Identification of knowdlege gaps and approach to the problem}
%\subsection{Spatiotemporal long-term variability in an almost isolated area}
%\subsection{Role of aerosols in the spatiotemporal varibility of the photovoltaic energy production}
%\subsection{Future availability of photovoltaic potential}


\chapter{State of knowdlege\label{cha:state}}
\section{Intermittency of PV}
\section{Variability sources in PV}
\section{From short to long term issues}
\section{Climate Change perspectives for solar resource and photovoltaic potential}

%\end{document}