% Este documento contiene el índice del manuscrito de la tesis.
%\documentclass[12pt,a4paper]{report}
%\usepackage{graphicx}
 
%\title{Introduction}
%\author{Claudia Guti\'errez}
%\date{ January 2018}


%\begin{document}

%\maketitle

%\tableofcontents{}

%%%%%%%%%%%%%%%%%%%%%%%%%%%%%%%%%%%%%%%%%%%%%%%%%%%%%%%%%%%%%%%%%%%%%%%%%%%%%%%%%%%%%%
\part{Introduction\label{cha:intro}}

%\section{Transversalidad}

% La ciencia aplicada o aplicación de la ciencia, ha evolucionado las sociedades a través de la conexión de los estudios de ciencia básica con los ingenios que implementaban el conocimiento estructural al día a día de los individuos. La construcción del puente entre los estudios fundamentales y su adaptación/evolución para el desarrollo de las sociedades, ha sido cuestión de una interpretación (y por lo tanto de intérpretes) capaz de econtrar el lenguaje adecuado para intercambiar este conocimiento.

%% Applied science has made societies evolved trough the linked between the basic science and the 'ingenios' that implement the structural knowdlege to the individual daylyligde. The bridge between the fundamental studies and its evolution or transformation into something applied for the societies development, has to be with the interpretation (so interpreters) that have been able to find the accurate languafe to exchange this knowdlege. %%

 % Esta transversalidad entre las disciplinas básicas de la física, la química o la biología con distintas acepciones prácticas, la mayoría de ellas recogidas bajo el paraguas de las ingenierías, desemboca ineludiblemente en una transformación de lo abstracto en lo tangible con la premisa o el ideal de mejorar el bienestar de las sociedades presentes y futuras. 

% Es sin embargo muy probable, que esta concepción de la aplicación científica haya desembocado en el mayor problema al que hacer frente desde.Siendo este antropocentrismo parte del problema, y no de la solución. Jorge Wasenberg escibe en su libro, ``el pensador intruso'' en referencia a la evolución de la ciencia y el progreso subyacente lo siguiente:

% ``Existen sobre todo dos vicios que tienden a inyectar ideología precocinada en la ciencia. Una de ellas se basa en las distintas formas de antropocentrismo y consiste en situar instintivamente al sujeto del conocimiento en el centro del cosmos. La historia del conocimiento es testigo: cada vez que barremos el Yo del centro del escenario el conocimiento avanza, y avanza sólo por ello.''

% Sin caer en lo pretencioso, este trabajo contribuye a la imbricación entre la ciencia del cambio climático y la tecnología renovable de producción eléctrica con mayor potencial a día de hoy, la solar fotovoltaica.

\chapter{Introduction}

\epigraphfontsize{\small\itshape}
\epigraph{''Begin at the beginning,'' the King said gravely, ''and go on till you
come to the end: then stop.''}{--- \textup{Lewis Carroll}, Alice in Wonderland}

\section{A changing world} 

Ideas a desarrollar:

* Un mundo en constante evolución.
* Cambio climático
* Transición energética global
* Cambios sociales/políticos.
* Migraciones

* Entrelazamiento de todas estas cuestiones.
* El conocimiento puro + el conocimiento aplicado: obligación de la ciencia a ser en la medida de lo posible aplicada a solucionar problemas de las distintas sociedades. Justicia social. Más si estos problemas han sido creados por el ser humano.


%Applied science has made societies evolved through the link between the basic science and the implementation of this structural knowdlege to the individual dailylife. The bridge between the fundamental studies and its evolution or transformation into something applied (useful) for the societies development is been a matter of interpretation or 'translation' (so interpreters/translator) to find the accurate language to exchange this knowdlege.

% Esta transversalidad entre las disciplinas básicas de la física, la química o la biología con distintas acepciones prácticas, la mayoría de ellas recogidas bajo el paraguas de las ingenierías, desemboca ineludiblemente en una transformación de lo abstracto en lo tangible con la premisa o el ideal de mejorar el bienestar de las sociedades presentes y futuras.

%Transversality between the fundamental disciplines of physiscs, chemistry or biology and their different applied branches, most of them under the umbrella of engenieering, end unavoidably into a transformation from the abstraction to the tanglible with the premise or ideal of improving wellbeing of present and future societies.

% Es sin embargo muy probable, que esta concepción de la aplicación científica haya desembocado en el mayor problema al que la humanidad tiene que hacer frente desde su existencia, el cambio climático. Siendo este antropocentrismo parte del problema, y no de la solución. Jorge Wasenberg escibe en su libro, ``el pensador intruso'' en referencia a la evolución de la ciencia y el progreso subyacente lo siguiente:

% Nevertheless, it is very likely that this conception of the scientific application had lead to the most challeging problem that humanity has to face from its existance, climate change, being this anthropocentrism part of the problem and not of the solution.

%In his book ``El pensador intruso'', Jorge Wasengber wrote about the evolution of science and the underlying progress what follows: 

%'There are above all two vices that tend to set 'pre-cooked' ideollogy into science. First is base on different kinds of anthropocentrism and consist in put the knowdlege subjet into the cosmos's center. The history of knowdlege is the witness: each time we 'barremos' the 'I' from the spot, the knowdlege progresses and only because of it.'
% ``Existen sobre todo dos vicios que tienden a inyectar ideología precocinada en la ciencia. Una de ellas se basa en las distintas formas de antropocentrismo y consiste en situar instintivamente al sujeto del conocimiento en el centro del cosmos. La historia del conocimiento es testigo: cada vez que barremos el Yo del centro del escenario el conocimiento avanza, y avanza sólo por ello.''

% Sin caer en lo pretencioso, este trabajo contribuye a la imbricación entre la ciencia del cambio climático y la tecnología renovable de producción eléctrica con mayor potencial a día de hoy, la solar fotovoltaica.

\section{Renewable Energies}

El concepto físico de energía se define como la capacidad que tiene un sistema para realizar un trabajo, clasificando las distintas formas de energía en dos grandes grupos: la energía cinética, relacionada con el movimiento del sistema, o la energía potencial, que atiende a la posición de mismo. En esta clasificación fundamental encontramos que la energía mecánica, electromagnética o térmica son formas de energía que se engloban dentro de la energía cinética, mientras que la energía química (de la cual es un subtipo la energía nuclear) o la energía potencial gravitacional (como es el caso de las centrales hidroeléctricas) se encontrarían dentro de la energía potencial.

%The physics concept of \textit{energy} is defined as the capacity of a system to perform a work and it is divided in two groups: kinetic energy, that is related to the movement of the system, and the potential energy, related to its position. For this fundamental classification, mechanical energy, electromagnetic and thermal energy are encompassed in the kinetic energy group, while the chemical energy or the gravitational potential energy, as it is the case of hydropower,  are inside the potential energy group.

Es elemental para nuestro estudio delimitar la diferencia entre formas y fuentes de energía. Las fuentes de energía son definidas como aquellas a partir de las cuales la energía 'útil', directamente aplicable para su finalidad (sea esta el movimiento, la producción de electricidad o los distintos procesos metabólicos en el ser humano), puede ser extraída directamente o mediante algun proceso de transformación.

% It is elementary to this work to delimit the differences between kinds and sources of energy. Sources of energy are defined as those from which useful energy can be extracted, either directly to is end (being this one movement, elecgricity production or matabolic processes) or through some transformation process.

En el contexto del consumo energético de nuestras sociedades, denominaremos fuentes de energía primaria a aquellas a partir de las cuales obtendremos la energía final, tras un proceso de extracción/transformación y transporte. Atendiendo a esto podemos considerar energía primaria a los combustibles fñosiles, la energía hidráulica, la energía solar o la biomasa. Estas fuentes de energía primaria proporcionarán la energía final que en muchas ocasiones será en forma de electricidad (salvo aquella destinada al transporte).

%Así, los combustibles fósiles son una fuente de energía, puesto que su energía 'útil' debe ser extraída. Es decir, extraemos la energía química contenida en los combustibles fósiles y necesitamos ciertos procesos de transformación entre la energía potencial de una masa de agua elevada que termina pasando a través de una turbina para obtener electricidad. Algunos ejemplos de fuentes de energía son los combustibles fósiles, la energía solar o la biomasa.

% It is basic in this work to delimit the difference between \textbf{energy kinds} and \textbf{energy sources}. The last are defined as those from where the \textit{useful} energy, directly applicable to its ends (electricity production, mechanical movement, metabolic processes...), can be extracted directly or by some transformation process. Thus, fossil fuels are a source of energy and some transformation are needed to extract the useful energy. 

Aparece de manera natural la clasificación de estas fuentes de energía primaria según su origen renovable, siendo estas últimas aquellas fuentes inagotables. El sol, a pesar de su indiscutible finitud,  es considerado una fuente inagotable de energía debido a la diferencia entre la escala temporal de la vida humana y la vida de la estrella a la que nos refereimos, muchos órdenes de magnitud mayor.

El uso de la energía renovable ha acompañado el desarrollo de la humanidad desde tiempos remotos. Desde el uso de la biomasa para generar energía térmica para calentarse, hasta la utilización de la energía eólica transformada en energía mecánica en los molinos tradicionales o en la navegación. Solo hacia la mitad del siglo 19, con la invención de la máquina de vapor, comenzó la utilización de los combustibles fósiles de manera masiva y con ello lo que se denominó la primera revolución industrial.

A pesar de constituir un periodo de gran desarrollo en términos tanto tecnológicos como de las sociedades y su bienestar, este lapso de tiempo es pequeño comparado con la historia de la humanidad. Además, este desarrollo, a pesar de haber conseguido que los niveles de vida alcancen los más altos estándares en nuestra historia, este hecho no ha sido ni mucho menos global ni 'isótropo' (homogéneo). 

El primer impulso para diversificar fuentes de energía primaria ocurrió a partir de la primera crisis del petróleo en los años 70 (ref). El embargo de los países productores de petróleo tuvo grandes consecuencias en los países importadores, lo que provocó que se comenzaran a considerar nuevas formas de energía para asegurar su estabilidad.

En las últimas décadas, además de factores socioeconómicos y geopolíticos que han llevado a la necesidad de limitar la dependencia del petróleo en los países importadores, el impulso para el crecimiento exponencial de estas tecnologías alternativas, ha venido provocado por el gran descenso en el precio de las tecnologías renovables y las políticas ``verdes'' impulsadas por los distintos organismos para combatir el cambio climático.

Crecimiento de la potencia instalada en el mundo. Crecimiento de la fotovoltaica. Descenso de precios de esta tecnología.

En 2018 se ha reportado que las renovables suponen un 55\% de la energía final consumida, con una importancia mayor en el sector de la caleffación y la climatización, así como en la electricidad. El transporte, en cambio, sigue siendo el sector con menor 'share' de renovables. 
% Gran crecimiento, consumo de combustibles fósiles sin oportunidad de reemplazo. Peak oil?
% Aunque las ernergías renovables se han utilizado desde 'siempre' (molinos), es a partir de los años 70 (crisis del petróleo) cuando vuelve a aparecer el interés por este tipo de energía.

% The evolution of new technologies based on renewable energies has been made in a context with two main drivers. In first place, a society concerned about environmental issues has to deal with 

% as well as interested in finding alternatives to traditional energy sources, Alternative energy generation sources have been evolving and have adquired an important role, mostly in the electricity sector, in the generation proccess
\subsection{Context of renewable energies}
\subsection{History and evolution}
\subsection{Scenarios and future evolution of renewable energies}
\subsection{Photovoltaic technology}
\section{Links between climate and renewable energy}



\section{Photovoltaic Energy}
%\section{Resource assessment}
\section{Climate change and the Mediterranean area}
\section{General scientific question: spatiotemporal behaviour of solar resource and photovoltaic production}
\section{Organization of the manuscript}
% \section{General scientific question: spatiotemporal variability of solar resource and photovoltaic production}
% \section{State of the art}
% \subsection{From short to long term varibility issues}
% \subsection{Solar radiation data measurements}
% \subsection{Satellite data}
% \subsection{Modelization of solar radiation}
% \section{Identification of knowdlege gaps and approach to the problem}
%\subsection{Spatiotemporal long-term variability in an almost isolated area}
%\subsection{Role of aerosols in the spatiotemporal varibility of the photovoltaic energy production}
%\subsection{Future availability of photovoltaic potential}


\chapter{State of knowdlege\label{cha:state}}
\section{Intermittency of PV}
\section{Variability sources in PV}
\section{From short to long term issues}
\section{Climate Change perspectives for solar resource and photovoltaic potential}

%\end{document}