\chapter*{Conclusiones\label{Conclusiones}}

Los resultados principales del trabajo muestran la amplitud en diferentes escalas temporales y espaciales de la variabilidad del recurso solar y de la productividad fotovoltaica. Los estudios de variabilidad en el largo plazo para la producción eléctrica, y cuya relevancia en la literatura es menor que la de los estudios de caracterización y predicción en el corto plazo, han mostrado tener la capacidad de abordar temas de interés para el desarrollo del sector enegético con alta penetración de energías renovables.

En los tres capítulos de resultados incluidos en este trabajo se investiga la respuesta de cada pregunta específica que emerge del problema científico principal. Estas tres cuestiones pueden resumirse así: la caracterización de la variabilidad interanual, la influencia de los aerosoles como factor determinante en la variabilidad espacio-temporal y la evolución de la producción fotovoltaica en condiciones de cambio climático. 

A pesar del evidente marco común para las tres cuestiones, cada una de ellas es discutida de manera individual, por lo que las principales conclusiones derivadas de los tres estudios se resumen a continuación de manera independiente.

\subsubsection{Análisis de variabilidad sobra la PI}

En el capítulo 5 se aplica un esquema de varias etapas basado en algoritmos de clustering para estudiar la variabilidad interanual de la producción fotovoltaica en la Península Ibérica. La Península Ibérica es considerada como una unidad eléctrica coherente dentro del sistema europeo, dado su caracter casi aislado desde el punto de vista físico, con la barrera natural de los Pirineos, y desde el punto de vista eléctrico, por las limitaciones en las interconexiones. Además, la gran variedad de climas en su relativamente pequeña extensión, hace que pueda ser considerada como un caso de estudio interesante. Las cuestiones científicas investigadas a través del esquema se resumen en tres:   

\begin{itemize}
  
\item  En primer lugar, se analiza la existencia de una distribución espacial óptima de clusters o regiones, capaz de explicar la variabilidad de la radiación solar en la PI.
\item En segundo lugar se estudian las principales características agrupadas por la distribución espacial, y analizan los cambios al considerar los distintos sistemas de seguimiento en los paneles PV.
\item Por último, se analiza si la complementariedad espacial entre las diferentes regiones puede ser estudiada a partir de este método.

\end{itemize}

El esquema que se presenta en el capítulo 5 permite caracterizar un área de acuerdo a la variabilidad del recurso solar y sistematiza la comparación entre diferentes sub-areas. Su flexibilidad permite ser aplicado a diferentes escalas temporales, diferentes recursos y diferentes areas.

La regionalización mediante clustering aplicada a la Península Ibérica permite distinguir las principales características climáticas de la zona y la variabilidad interanual del recurso y la producción pueden ser analizadas y comparadas espacialmente. En general, el CV de la productividad anual de los diferentes clusters a lo largo de la Península varía entre muy poca variabilidad, 2$\%$ hasta alrededor de un 5$\%$, lo que significa un recurso bastante estable en esas escalas temporales, aunque las diferencias entre áreas son claras, y la variabilidad interanual de las series mensuales más alta, por encima de un 20$\%$ para algunas zonas en los meses de invierno. El CV crece con el cambio en el sistema de seguimiento fotovoltaico, siendo el seguidor a doble eje más sensible a cambios en la radiación solar y por tanto las diferencias entre clusters mayores.

Este capítulo muestra como existe una configuración de regiones que se define por las diferentes características de la variabilidad sobre la IP, lo que puede ser útil para un posterior análisis más profundo. Además, este análisis espacial permite investigar diferentes escalas temporales donde la complementariedad entre zonas pueda surgir.

\subsubsection{Impcato de los aerosoles en la producción fotovoltaica en el área Euro-Mediterránea}

Los resultados del capítulo 7 buscan cuantificar el impacto de los aerosoles en la variabilidad espacio-temporal de la radiación solar y la producción fotovoltaica en escalas climáticas. De manera concreta, en este capítulo encontramos la respuesta a las siguientes tres preguntas:

\begin{itemize}
\item En primer lugar se busca conocer si los aerosoles tienen un impacto significativo en la producción fotovoltaica en el área Euro-Mediterránea.
\item En segundo lugar se busca cuantificar este impacto en la producción fotovoltaica en escalas que abarcan desde estacionales hasta decadales, en condiciones de clima presente.
\item Por último, de manera transversal podemos evaluar si el uso de modelos climáticos para estudios de evaluación de recurso es adecuado.
\end{itemize}

Las tendencias pasadas observadas en la radiación solar sobre Europa, han sido simuladas con un modelo regional climático. Sólo cuando la tendencia decreciente de los sulfatos es considerada en las simulaciones, el modelo es capaz de reproducir el incremento de la radiación observado en Europa desde los años 80.

La simulación multi-decadal que representa la tendencia en la radiación observada es utilizada para cuantificar el impacto de la misma en una planta de generación fotovoltaica. Los resultados muestran que el periodo de 'brillantez' en Europa, hubiera supuesto un incremento anual en la producción de más de un 10$\%$ en algunas áreas de Europa Central, lo que significa un gran impacto para una hipotética planta de generación en esta zona.

A partir de un estudio de sensibilidad se cuantifica el impacto de los aerosoles para la escala interanual y estacional. Aunque la variabilidad no es importante en escala interanual, sí se observa una alta variabilidad espacio-temporal en la escala estacional. 

%{\color{red} Past SSR over Europe simulated with a regional climate model is only able to follow the observed trend if an aerosol dataset is used to this purpose. In this context, these simulations are used in order to quantify the impact that these trends would have in a simulated PV power plant. The results show that the brightenning period over Europe would lead to an increase in yearly production of more than 10$\%$ in some areas of Central Europe. This means a big impact in a potential PV project over the area. Also, seasonal variability of PV productivity in actual climate conditions have a wide impact accross the area.}

El valor añadido de las simulaciones que incluyen los aerosoles se demuestra mediante el test de sensibilidad aplicado en este capítulo. A pesar de las limitaciones de los RCMs, se ilustra su capacidad como herramienta para investigar las interaciones entre radiación solar y aerosoles y sus consecuencias no solo en el clima, sino en la radiacion como recurso de energia solar.

%in this chapter illustrates that regional climate models are useful to these kind of experiments, despite of the limitations discussed in previous sections, in order to help in the knowdlege of solar radiation and aerosols interactions. The added value of simulations including aerosols dataset for the estimation of PV production is clear for the whole domain under study. Also it is shown that PV production data can be reproduce using simulations with aerosols for at least two real PV plants location.

Es importante remarcar la dificultad de obtener datos reales de plantas de generación eléctrica, en concreto de plantas de generación fotovoltaica, para la validación completa de la cadena de modelado utilizada en este capítulo. Sin embargo, la comparación de la producción obtenida con la de algunas plantas de producción disponibles en escala local, apuntan el impacto positivo de la consideración de información detallada sobre aerosoles en las simulaciones con RCMs. 

%Para reducir la incertidumbre de las estimaciones de PV, es posible reducir el sesgo de la radiación que proviene de las simulaciones climáticas a partir de metodologías de corrección de sesgo. Sin embargo, en nuestro caso, esto podría enmascarar los efectos de la inclusión de aerosoles, siendo su cuantificación y determinación el fin último del trabajo. Es, por otro lado, remarcable como la inclusión de los aerosoles en las simulaciones climáticas mejora bastante la radiación.

% In order to go further in the use of RCMs for energy resource assessment, being able to compare with real data is one of the 'bottle necks'. However, in order to reduce uncertainty in the PV estimations, it is possible to reduce solar irradiation bias. In first place, we have seen (and it has been shown by other authors) that inclusion of aerosols highly improve solar irradiation at the surface but also, some bias correction can be forward applied and evaluated in order to be used as input of the modelling chain.  


\subsubsection{Future projections of PV potential under climate change scenarios}

La creciente preocupación acerca de la disponibilidad de los recursos renovables bajo condiciones de cambio climático, ha motivado los recientes estudios de modelos climáticos para evaluar los posibles cambios en los diferentes recursos.

La radiación solar en superficie bajo condiciones de cambio climático, ha sido investigada en diferentes trabajos, así como las proyecciones futuras de potencial fotovoltaico. Sin embargo, la cantidad de estudios dedicados al futuro de los recursos renovables es aún limitado.

La discrepancia entre los modelos climáticos globales y regionales en las proyecciones de SSR sobre Europa es un tema que merece ser considerado para su estudio. En el capítulo 7, la relación entre diferentes proyecciones climáticas y la representación que cada uno de los modelos hace de los aerosoles es unvestigada para responder a las siguientes preguntas:

\begin{itemize}
\item Debido a la importancia que los posibles cambios en el recurso solar tienen para las actividades de planifiación en el sector de la energía solar, ¿cómo son las proyeciones de potencial futuro de PV sobre la zona Euro-Mediterránea para los escenarios de cambio climático?
\item ¿Es importante el papel de los aerosoles y su evOlución para entender las discrepancias entre los modelos globales, GCMs y los modelos regionales RCMs en esa zona?
\end{itemize}

Las simulaciones regionales de clima del ensemble de EURO-CORDEX se usan para el cálculo del potencial PV. Los resultados muestran que incluir la evolución temporal de los aerosoles en las proyecciones es clave para poder entender los cambiosproyectados en la producción PV. Las simulaciones de RCMs que incluyen esta evolución temporal, reproducen el incremento en la productividad fotovoltaica sobre Europa que ha sido proyectada con los modelos globales, mostrando un signo positivo en la anomalía de radiación, al contrario que el resto de RCMs.

Este resultado difiere de los anteriores estudios usando RCMs porque se enfoca a la selección de las simulaciones en función de su representación de aerosoles, descartando el estudio desde el punto de vista del ensemble.

Los resultados muestran que para las proyecciones que incluyen aerosoles, el cambio en el potencial PV para la mitad del siglo XXI presenta valores altos para la zona de Europa-Central y el Sur de Europa. Este patrón está relacionado con los escenarios de aerosoles que proyectan un descenso de los aerosoles de origen antropogénico en estas zonas.

La magnitud del cambio varía de un modelo a otro, un resultado que es importante no solo por la cautela con la que deben tomarse estos resultados para dar mensajes a la industria solar, sino también porque señala la manera de proceder en trabajos futuros que deben cuantificar el impacto del forzamiento de aerosoles para cada modelo. En este aspecto, el FPS (flagship pilot study) que trabaja dentro del marco de EURO-CORDEX es un ejercicio que sistematiza las simulaciones de RCMs aplicando distintos estudios de sensibilidad a cada modelo en función los aerosoles empleados. Este trabajo ayudará a comprender mejor como la evolución de los aerosoles va a afectar no solo a la radiación, sino al sistema climático.

Esta disertación comprende un conjunto de estudios que tratan problemas relacionados con la variabilidad en el largo plazo para la radiación solar y para la producción fotovoltaica que se enmarcan dentro del contexto climático. El acercamiento a estos problemas en cada uno de los capítulos, crea una metodología aplicable a distintos problemas y sienta las bases para la elaboración de estudios futuros, tanto en la evaluación del impacto de los aerosoles en otras escalas temporales, como en la aplicación de las metodologías de clustering para el análisis espacio-temporal de los recursos y la producción renovables. 