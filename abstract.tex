%\begin{summary}
\chapter*{Abstract\label{cha:abstract}}
%  \begin{abstract}
  The ongoing energy transition and the growth of the share of renewables technologies in the energy system, especially in the electricity mix generation, have risen concern about the spatio-temporal characteristics of the resources. This knowledge demand from the involved stakeholders, has increased not only for the short-term time scales, but also in the climatic scales that affect to different stages of the renewables deployment and development. The interest about the availability of renewable resources under climate change conditions is especially relevant. The possible changes in the resource amount or its variability might affect the planning and future operation activities in power plants that are being projected at this time. 

The present work is based on the study of solar resource and photovoltaic production over the Euro-Mediterranean area with a climatic perspective. The variability issue, that makes most of the renewable technologies not available by demand, is addressed from three different perspectives. At the same time, these three approaches give answers to concrete scientific questions in each of the results chapters.

In the first place, the interannual variability and complementarity of solar resource and photovoltaic productivity in the Iberian Peninsula is analysed using a multi-step scheme that includes a regionalization through clustering algorithms. The method allows to systematize intercomparison among zones inside the region studied.

Secondly, the role of aerosols in the spatio-temporal variability of the photovoltaic production is analysed for the Euro-Mediterranean area, which is highly influenced by aerosols from different sources.

Finally, future projections of photovoltaic energy are analysed under climate change conditions over Europe.

The three studies show that climate time-scales are also relevant in terms of solar resource and photovoltaic productivity and deserve attention for a better integration of photovoltaic energy in the energy system.
%\end{abstract}

%\end{summary}