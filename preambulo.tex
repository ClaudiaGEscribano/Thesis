\usepackage[T1]{fontenc}
\usepackage[utf8]{inputenc}
\usepackage[b5paper]{geometry}%[a4paper]{geometry}
%\geometry{verbose,tmargin=2.5cm,bmargin=2.5cm,lmargin=2.5cm,rmargin=2.5cm}
%\usepackage{geometry}
%\geometry{
% b5paper
% total={170mm,257mm},
% left=20mm,
% top=20mm,
% }
\pagestyle{Ruled}
%\settypeblocksize{6.5in}{4.5in}{*}
%\setlrmargins{*}{*}{1}
%\checkandfixthelayout
\usepackage{array}
\usepackage{verbatim}
\usepackage{prettyref}
\usepackage{booktabs}
\usepackage{textcomp}
\usepackage{url}
\usepackage{amsmath}
\usepackage{chemarr}%flechas para reacciones químicas (SFER.tex)
\usepackage{graphicx}
\usepackage{amssymb}
\usepackage{nomencl}
\usepackage[usenames,dvipsnames]{xcolor}

%\includeonly{DataMethods}

%\OnehalfSpacing

% the following is useful when we have the old nomencl.sty package
% \providecommand{\printnomenclature}{\printglossary}
% \providecommand{\makenomenclature}{\makeglossary}
\makenomenclature

\usepackage[caption=false]{subfig}
%Configuración de los caption
%\PassOptionsToPackage{caption=false}{subfig}%Evita que el paquete subfig lo descabale todo
\captiontitlefont{\itshape}
\captionnamefont{\scshape}
%\captionstyle{\centering}
\hangcaption


%\usepackage[spanish]{babel}
%\addto\shorthandsspanish{\spanishdeactivate{~<>}}
 

\usepackage[emulate=units]{siunitx}
\sisetup{per=fraction, fraction=nice}
\newunit{\wattpeak}{Wp}
\newunit{\watthour}{Wh}
\newunit{\watt}{W}
\newunit{\metro}{m}

% \usepackage{lscape}
\usepackage{mathpazo}%Letra palatino con fuentes para matemáticas
\usepackage{flafter}%obliga a que los flotantes aparezcan después de su referencia
\usepackage{memhfixc}
\usepackage{mempatch}


\raggedbottom
\sloppybottom
\clubpenalty=10000
\widowpenalty=10000

%\raggedbottomsection
\feetbelowfloat

% !! esta seccion esta comentada en veleta

\usepackage[citestyle=alphabetic, bibstyle=alphabetic, maxbibnames=5, minbibnames=3, backend=bibtex, doi=true, url=true]{biblatex}

\DefineBibliographyStrings{spanish}{%
  andothers        = {et\addabbrvspace al\adddot},
  andmore          = {et\addabbrvspace al\adddot},
  in               = {},
}

% \addbibresource{../biblio.bib}
\addbibresource{Data.bib}

\let\cite\parencite

\renewcommand{\bibsection}{%
	\chapter*{\bibname}
	\bibmark
	\phantomsection
	\addcontentsline{toc}{chapter}{\bibname}
	\prebibhook}
\renewcommand{\bbltechreport}{Informe T\'ecnico}

% !!

%definition of some colors I could use.      
\definecolor{amaranth}{rgb}{0.9, 0.17, 0.31}
\definecolor{brightmaroon}{rgb}{0.76, 0.13, 0.28}

\usepackage{hyperref}


\hypersetup{
    bookmarks=true,         % show bookmarks bar?
    unicode=true,          % non-Latin characters in Acrobat’s bookmarks
    bookmarksnumbered=false,
    bookmarksopen=false,
    breaklinks=true,
    backref=true,
    pdftoolbar=true,        % show Acrobat’s toolbar?
    pdfmenubar=true,        % show Acrobat’s menu?
    pdffitwindow=false,     % window fit to page when opened
    pdfstartview={FitH},    % fits the width of the page to the window
    pdftitle={Tesis Claudia},    % title
    pdfauthor={Claudia Gutiérrez},     % author
    pdfsubject={Energia Solar Fotovoltaica},   % subject of the document
    pdfcreator={AucTeX/Emacs},   % creator of the document
    pdfproducer={LaTeX}, % producer of the document
    pdfkeywords={radiación solar, energía solar fotovoltaica, energías
    renovables}, % list of keywords
    pdfnewwindow=true,      % links in new window
    pdfborder={0 0 0},
    colorlinks=true,       % false: boxed links; true: colored links
    linkcolor=grey,          % color of internal links
    citecolor=brigthmaroon %BrickRed,        % color of links to bibliography
    filecolor=black,      % color of file links
    urlcolor=Blue           % color of external links 
}
	

\DeclareSIUnit\kWh{kWh}
\DeclareSIUnit\Wh{Wh}
\DeclareSIUnit\Wp{Wp}
\DeclareSIUnit\kWp{kWp}
\DeclareSIUnit\amperehour{Ah}
\DeclareSIUnit\celula{celula}

%\spanishdecimal{.} %Para que no lo sustituya automáticamente por comas
\addto\captionsspanish{%
\def\tablename{Tabla}%
\def\listtablename{\'Indice de tablas}}

\renewcommand\nomname{Nomenclatura}
\def\nompreamble{\addcontentsline{toc}{chapter}{\nomname}\markboth{\nomname}{\nomname}}


%\@addtoreset{equation}{section}
%\renewcommand{\theequation}{\thesection.\arabic{equation}}
%\numberwithin{equation}{section}
%\@addtoreset{table}{section}
%\renewcommand{\thetable}{\thesection.\arabic{table}}
%\numberwithin{table}{section}
%\@addtoreset{figure}{section}
%\renewcommand{\thefigure}{\thesection.\arabic{figure}}
%\numberwithin{figure}{section}


%\declarebtxcommands{spanish}{%
 % \def\btxphdthesis#1{\protect\foreignlanguage{spanish}{Tesis Doctoral}}%
%}
%\setbibliographyfont{lastname}{\scshape}%Pone los autores en Small Caps



%Configuración de MEMOIR
%%Pone la fecha en SMALL CAPS y hacia la derecha
%%pagina 60 de memman.pdf
\pretitle{\vfill \begin{center} \bfseries\HUGE \color{Black}}% \scshape \HUGE \color{Black}}
\posttitle{\par\end{flushright}}

\preauthor{\begin{center} \large}% \scshape}
\postauthor{\par\end{flushright}}

\date{}
\predate{\vfill \begin{flushright}\large\scshape}
\postdate{\par\end{flushright}\vfill}

\setsecnumdepth{subsection}


%\definecolor{ared}{rgb}{.647,.129,.149}
%\renewcommand{\colorchapnum}{\color{ared}}
%\renewcommand{\colorchaptitle}{\color{ared}}
%\chapterstyle{pedersen}
\chapterstyle{ger}

\setlength{\afterchapskip}{35pt}
\maxtocdepth{section}

% \setcounter{topnumber}{3}
%\setcounter{bottomnumber}{2}
%\setcounter{totalnumber}{4}
\renewcommand{\topfraction}{0.85}
\renewcommand{\bottomfraction}{0.5}
\renewcommand{\textfraction}{0.15}
\renewcommand{\floatpagefraction}{0.7}


%Centra las figuras en los flotantes y los enmarca
\makeatletter
\renewenvironment{figure}[1][]{%
     	\@float{figure}%
		%\begin{framed}    
		\precaption{\rule{\linewidth}{0.4pt}\par}%En las figuras el caption va debajo
		%\hrule\vspace{\onelineskip}
		\centering
		  }{%
		%\end{framed}
		%\postcaption{\rule{\linewidth}{0.4pt}}
		%\vspace{\onelineskip}\hrule
    	\end@float	
}
\makeatother

\makeatletter
\renewenvironment{table}[1][]{%
      	\@float{table}%
		%\begin{framed}    
		\postcaption{\rule{\linewidth}{0.4pt}\par}%En las tablas el caption va encima
		\centering
		  }{%
		%\end{framed}
    	\end@float	
}
\makeatother


\renewcommand{\textfloatsep}{10pt}%Espacio entre el flotante y el texto

%backgroung image
\usepackage{eso-pic} 

%http://stackoverflow.com/questions/240097/how-to-create-a-background-image-on-titlepage-with-latex
\newcommand\BackgroundPic{
  \put(350,-150){
    \parbox[b][0.5\paperheight]{0.5\paperwidth}{%
      \includegraphics[scale=0.5]{../figs/johnny_automatic_old_sun}%
}}}

\newcommand\BackgroundPicLight{
  \put(350,-150){
    \parbox[b][0.5\paperheight]{0.5\paperwidth}{%
 %     \vfill 
%\centering
      \includegraphics[scale=0.5]{../figs/johnny_automatic_old_sun_light}%
%\vfill
}}}

%claudia
\renewenvironment{abstract}%
{\cleardoublepage\null \vfill\begin{center}%
\bfseries\abstractname\end{center}}%
{\vfill\null}

%%%% claudia:
% \usepackage{color,calc,graphicx,soul,fourier}
% \definecolor{nicered}{rgb}{.647,.129,.149}
% \makeatletter
% \newlength\dlf@normtxtw
% \setlength\dlf@normtxtw{\textwidth}
% \def\myhelvetfont{\def\sfdefault{mdput}}
% \newsavebox{\feline@chapter}
% \newcommand\feline@chapter@marker[1][4cm]{%
% \sbox\feline@chapter{%
% \resizebox{!}{#1}{\fboxsep=1pt%
% \colorbox{nicered}{\color{white}\bfseries\sffamily\thechapter}%
% }}%
% \rotatebox{90}{%
% \resizebox{%
% \heightof{\usebox{\feline@chapter}}+\depthof{\usebox{\feline@chapter}}}%
% {!}{\scshape\so\@chapapp}}\quad%
% \raisebox{\depthof{\usebox{\feline@chapter}}}{\usebox{\feline@chapter}}%
% }
% \newcommand\feline@chm[1][4cm]{%
% \sbox\feline@chapter{\feline@chapter@marker[#1]}%
% \makebox[0pt][l]{% aka \rlap
% \makebox[1cm][r]{\usebox\feline@chapter}%
% }}
% \makechapterstyle{daleif1}{
% \renewcommand\chapnamefont{\normalfont\Large\scshape\raggedleft\so}
% \renewcommand\chaptitlefont{\normalfont\huge\bfseries\scshape\color{nicered}}
% \renewcommand\chapternamenum{}
% \renewcommand\printchaptername{}
% \renewcommand\printchapternum{\null\hfill\feline@chm[2.5cm]\par}
% \renewcommand\afterchapternum{\par\vskip\midchapskip}
% \renewcommand\printchaptertitle[1]{\chaptitlefont\raggedleft ##1\par}
% }
% \makeatother
% \chapterstyle{daleif1}
%probar bluebox

% \usepackage{calc,color}
% \newif\ifNoChapNumber
% \newcommand\Vlines{%
% \def\VL{\rule[-2cm]{1pt}{5cm}\hspace{1mm}\relax}
% \VL\VL\VL\VL\VL\VL\VL}
% \makeatletter
% \setlength\midchapskip{0pt}
% \makechapterstyle{VZ43}{
% \renewcommand\chapternamenum{}
% \renewcommand\printchaptername{}
% \renewcommand\printchapternum{}
% \renewcommand\chapnumfont{\Huge\bfseries\centering}
% \renewcommand\chaptitlefont{\Huge\bfseries\raggedright}
% \renewcommand\printchaptertitle[1]{%
% \Vlines\hspace*{-2em}%
% \begin{tabular}{@{}p{1cm} p{\textwidth-3cm}}%
% \ifNoChapNumber\relax\else%
% \colorbox{black}{\color{white}%
% \makebox[.8cm]{\chapnumfont\strut \thechapter}}
% \fi
% & \chaptitlefont ##1
% \end{tabular}
% \NoChapNumberfalse
% }
% \renewcommand\printchapternonum{\NoChapNumbertrue}
% }
% \makeatother
% \chapterstyle{VZ43}

\usepackage{fourier} % or what ever
\usepackage[scaled=.92]{helvet}%. Sans serif - Helvetica
\usepackage{color,calc}
\newsavebox{\ChpNumBox}
\definecolor{ChapBlue}{rgb}{0.00,0.65,0.65}
\makeatletter
\newcommand*{\thickhrulefill}{%
\leavevmode\leaders\hrule height 1\p@ \hfill \kern \z@}
\newcommand*\BuildChpNum[2]{%
\begin{tabular}[t]{@{}c@{}}
\makebox[0pt][c]{#1\strut} \\[.5ex]
\colorbox{ChapBlue}{%
\rule[-10em]{0pt}{0pt}%
\rule{1ex}{0pt}\color{black}#2\strut
\rule{1ex}{0pt}}%
\end{tabular}}
\makechapterstyle{BlueBox}{%
\renewcommand{\chapnamefont}{\large\scshape}
\renewcommand{\chapnumfont}{\Huge\bfseries}
\renewcommand{\chaptitlefont}{\raggedright\Huge\bfseries}
\setlength{\beforechapskip}{20pt}
\setlength{\midchapskip}{26pt}
\setlength{\afterchapskip}{40pt}
\renewcommand{\printchaptername}{}
\renewcommand{\chapternamenum}{}
\renewcommand{\printchapternum}{%
\sbox{\ChpNumBox}{%
\BuildChpNum{\chapnamefont\@chapapp}%
{\chapnumfont\thechapter}}}
\renewcommand{\printchapternonum}{%
\sbox{\ChpNumBox}{%
\BuildChpNum{\chapnamefont\vphantom{\@chapapp}}%
{\chapnumfont\hphantom{\thechapter}}}}
\renewcommand{\afterchapternum}{}
\renewcommand{\printchaptertitle}[1]{%
\usebox{\ChpNumBox}\hfill
\parbox[t]{\hsize-\wd\ChpNumBox-1em}{%
\vspace{\midchapskip}%
\thickhrulefill\par
\chaptitlefont ##1\par}}%
}
\chapterstyle{BlueBox}

\epigraphfontsize{\small\itshape}
\setlength\epigraphwidth{8cm}
\setlength\epigraphrule{0pt}
