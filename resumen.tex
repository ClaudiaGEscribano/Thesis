\chapter*{Resumen\label{cha:resumen}}

La presente tesis doctoral supone un acercamiento al estudio de los recursos renovables en escalas temporales climáticas. La transición energética en marcha, y el crecimiento de la participación de las tecnologías renovables en el sistema energético y especialmente en el mix de generación eléctrico, han aumentado el interés por conocer las características espacio-temporales de los recursos. Esta demanda de conocimiento por parte de los sectores involucrados en la energía, ha crecido no sólo en las escalas temporales más cortas, sino en su escalas temporales denominadas climáticas, que afectan a distintas etapas del desarrollo de las plantas de generación.

Especialmente relevante es el interés sobre la disponibilidad de los recursos bajo condiciones de cambio climático, los posibles cambios en su cantidad o variabilidad pueden afectar a la planificación futura y a la operación de plantas que sean proyectadas en el momento en el que nos encontramos.

El presente manuscrito aborda el problema de variabilidad en escalas climáticas desde tres perspectivas distintas que a su vez dan respuesta a preguntas concretas en cada uno de los capítulos destinados a los resultados. Estos tres puntos pueden resumirse de la siguiente manera:

\begin{enumerate}
\item Estudio de la variabilidad de la producción fotovoltaica en la Península Ibérica mediante metodologías de clustering
\item Análisis del impacto de los aerosoles en la variabilidad espacio-temporal de la producción fotovoltaica en el area Euro-Mediterránea
\item Proyecciones futuras de potencial fotovoltaico bajo condiciones de cambio climático en Europa.
\end{enumerate}

El primer estudio \cite*{Gutierrez2017}, propone un método para analizar la variabilidad y la complementariedad del recurso solar y de la productividad fotovoltaica. El empleo de técnicas de clustering sobre el área de estudio ha permitido encontrar una partición óptima que facilita el análisis espacial, especialmente el de complementariedad. Un modelo paramétrico de producción fotovoltaica es utilizado para obtener la producción potencial en cada punto.

La zona de estudio presenta una variabilidad interanual baja, lo que la hace especialmente relevante para el desarrollo e integración masiva de tecnologías solares, existiendo aún diferencias entre las disintas zonas. De hecho, se ha encontrado cierto grado de complementariedad entre ellas, que puede ayudar a la compensación espacial cuando la disponibilidad del recurso sea baja. Aún así, este estudio puede servir de base para estudiar la complentariedad con otros recursos como el eólico, lo que facilitaría la gestión de la producción en un sistema con alta penetración de ambas tecnologías.

Dentro de los distintos factores que originan la variabilidad tanto del recurso solar por un lado, como de la producción fotovoltaica como consecuenciade la primera, el papel de los aerosoles es analizado mediante una cadena de modelado utilizando simulaciones climáticas y el modelo fotovoltaico anterior en el segundo punto. Mediante el diseño de un test de sensibilidad determinamos la variabilidad espacio-temporal consecuencia de los aerosoles en el área Euro-Mediterránea.

Los resultados \cite*{Gutierrez2018} muestran que influencia de los aerosoles en el patron espacial, el ciclo estacional y las tendencias de largo plazo de la producción es importante. La sensibilidad de la producción anual es alta en en la zona de Europa central y el tipo de seguidor del sistema fotovoltaico considerado es importante en el cálculo.

En este aspecto, se concluye que los aerosoles no pueden despreciarse en la producción en escalas temporales largas sobre la zona Euro-Mediterránea. Además, el crecimiento potencial debido a la posible reducción de aerosoles antropogénicos se muestra mediante la simulación del periodo de 'brightening' sobre Europa, ilustrando la posible evolución de las zonas de alta contaminación.

Los resultados de este punto muestran además la utilidad los modelos regionales para estudios de sensibilidad y atribución concretos que pueden ayudar a entender mejor la variabilidad espacio-temporal de los recursos renovables.

El tercer problema estudiado en este documento se centra en las proyecciones futuras en condiciones de cambio climático. La zona Euro-Mediterránea es evaluada para determinar los potenciales cambios en la radiación solar como recurso de la energía fotovoltaica. La influencia de los aerosoles, como factor determinante en las proyecciones propuestas con diferentes modelos climáticos regionales, es analizada con el objeto de determinar su papel en la productividad futura.

Para abordar este tercer punto, se parte de la observación de que las anomalías de radiación proyectadas en la zona por los modelos globales, GCM, son de signo opuesto a la mayoría de las proyecciones de los modelos regionales, RCM. En este aspecto, se ha analizado el papel de los aerosoles en las simulaciones de los modelos regionales como un factor que puede influir en el signo de la anomalía de radiación solar.

Los resultados muestran que las proyecciones de los modelos regionales que incluyen la evolución tempoarl de aerosoles en los escenarios, coinciden en el signo con la anomalía proyectada por los modelos regionales. 

Los resultados de este punto, suponen información relevante para la transición energética en marcha en muchos de los países de la zona, así como para el desarrollo de los servicios climáticos cada vez más presentes.

