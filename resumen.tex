\chapter*{Resumen\label{cha:resumen}}

\subsection{Introducci\'on y motivaci\'on}

La transición energética en marcha y el crecimiento de la participación de las tecnologías renovables en el sistema energético, especialmente en el mix de generación eléctrico, han aumentado el interés por conocer las características espacio-temporales de los distintos recursos renovables. Esta demanda de conocimiento por parte de los sectores involucrados, ha crecido no sólo en las escalas temporales más cortas, necesarias en la operación y gestión de las plantas, sino también en las escalas temporales denominadas climáticas, que afectan a distintas etapas de su desarrollo como la planificación y la financiación. Es en el estudio y caracterización de estas escalas climáticas en lo que se sustenta la presente tesis doctoral.

Especialmente relevante es el creciente interés sobre la disponibilidad de los recursos bajo condiciones de cambio climático, los posibles cambios en su cantidad o variabilidad pueden afectar a la planificación futura y a la operación de plantas que sean proyectadas en el momento en el que nos encontramos. De igual manera, los resultados obtenidos pueden proporcionar información relevante para la elaboración de políticas orientadas al aumento de la participación renovable en el sector energético.

\subsection{Objetivos}

El presente trabajo aborda el problema de variabilidad en escalas climáticas desde tres perspectivas distintas, dando a su vez respuesta a preguntas concretas en cada uno de los capítulos destinados a los resultados. Estos tres puntos pueden resumirse de la siguiente manera:

\begin{enumerate}
\item Estudio de la variabilidad interanual de la producción fotovoltaica y de su complementariedad en la Península Ibérica a través de metodologías de clustering.
\item Cuantificación del impacto de los aerosoles en la variabilidad espacio-temporal de la productividad fotovoltaica en el area Euro-Mediterránea.
\item Evolución de las proyecciones de potencial fotovoltaico bajo condiciones de cambio climático en Europa.
\end{enumerate}

\subsection{Resultados}

El primer estudio \cite*{Gutierrez2017}, propone un método para analizar la variabilidad y complementariedad del recurso solar y de la productividad fotovoltaica en la Península Ibérica. El empleo de técnicas de clustering sobre la zona de estudio ha permitido encontrar una partición óptima de regiones (o clusters) a partir de la variabilidad, lo que facilita el análisis espacial, especialmente el de complementariedad. Un modelo paramétrico de producción fotovoltaica es utilizado para obtener la producción potencial en cada punto del dominio estudiado.

La zona de estudio presenta una variabilidad interanual baja, lo que la hace especialmente relevante para el desarrollo e integración masiva de tecnologías solares, existiendo aún así diferencias entre las disintas zonas. Además, se ha encontrado cierto grado de complementariedad entre ellas, lo que podría ayudar a la compensación espacial cuando la disponibilidad del recurso sea baja. Este estudio puede servir de base para futuros trabajos que estudien la complentariedad con otros recursos como el eólico, lo que facilitaría la gestión de la producción en un sistema con alta penetración de ambas tecnologías.\\

\starbreak

En segundo lugar, dentro de los factores que originan la variabilidad tanto del recurso solar por un lado, como de la producción fotovoltaica por otro, el papel de los aerosoles es analizado mediante una cadena de modelado utilizando simulaciones climáticas y el modelo fotovoltaico param\'etrico. A pesar de que la nubosidad es el factor que más impacta normalmente en la producción fotovoltaica, reduciendo la radiación directa que llega al generador, el impacto de los aerosoles puede ser muy alto en algunas zonas. Mediante el diseño de un test de sensibilidad, determinamos la variabilidad espacio-temporal que es consecuencia de los aerosoles en el área Euro-Mediterránea.

Los resultados \cite*{Gutierrez2018} muestran una influencia importante de los aerosoles en el patrón espacial, el ciclo estacional y las tendencias de largo plazo de la producción. La sensibilidad de la producción anual es alta en en la zona de Europa central y el tipo de seguidor del sistema fotovoltaico considerado es relevante en el cálculo.

En este aspecto, se concluye que los aerosoles no pueden despreciarse en la producción en escalas temporales largas. Además, el crecimiento potencial debido a una reducción de aerosoles antropogénicos se muestra mediante la simulación del periodo de 'brightening' ocurrido a partir de los años 80 en Europa. Los resultados ilustran la posible evolución de otras zonas con alta contaminación y el potencial aumento en el recurso y la producción fotovoltaica.

Los resultados de este punto muestran además la utilidad los modelos regionales para estudios de sensibilidad y atribución concretos que pueden ayudar a entender mejor la variabilidad espacio-temporal de los recursos renovables.\\

\starbreak

El tercer problema estudiado en este documento se centra en las proyecciones futuras en condiciones de cambio climático. La zona Euro-Mediterránea es evaluada para determinar los potenciales cambios en la radiación solar como recurso de la energía fotovoltaica y el impacto en la producción como consecuencia de ello. La influencia de los aerosoles, como factor determinante en las proyecciones propuestas con diferentes modelos climáticos regionales, es analizada con el objeto de determinar su papel en la productividad futura.

Para abordar este tercer punto, se parte del hecho de que las anomalías de radiación proyectadas en Europa por los modelos globales, GCMs, son de signo opuesto a la mayoría de las proyecciones de los modelos regionales, RCMs. En este aspecto, se ha analizado el papel de los aerosoles en las simulaciones de los modelos regionales como un factor determinante en el cambio proyectado de radiación solar.

Los resultados muestran que las proyecciones de los modelos regionales que incluyen la evolución temporal de aerosoles en los escenarios, coinciden en el signo con la anomalía proyectada por los modelos globales, es decir, proyectan un aumento del potencial fotovoltaico en Europa. La magnitud del cambio proyectado depende del modelo, con valores de mas de un 10$\%$ para la productividad en verano para al menos uno de los modelos en la zona de Europa Central, lo que supone una informaci\'on importante para los pa\'ises involucrados.

Estos resultados suponen información relevante para la transición energética en marcha en muchos de los países de la zona, así como para el desarrollo de los servicios climáticos cada vez más presentes.

\nomenclature[GCMs]{$GCMs$}{Global Climate Models}
\nomenclature[RCMs]{$RCMs$}{Regional Climate Models}